\pagenumbering{roman}
\setcounter{page}{1}

\selecthungarian

%----------------------------------------------------------------------------
% Abstract in Hungarian
%----------------------------------------------------------------------------
\chapter*{Kivonat}\addcontentsline{toc}{chapter}{Kivonat}

Az elmúlt évek rohamos technológiai fejlődésének köszönhetően az okostelefonok hatalmas teret hódítottak maguknak, társadalmunk jelentős hányada mára már rendelkezik legalább egy ilyen eszközzel. Ennek következményeképpen szinte mindent készülékeinken intézünk: kapcsolattartást, hivatalos ügyeket, vagy éppen a tanulást. Ezek megkönnyítésére folyamatosan jelennek meg a különböző natív alkalmazások, melyeket feltelepítve csupán pár kattintásra egyszerűsödnek az elvégezni kívánt feladatok. 

Ám rengeteg alkalmazási területen még nincs igazán jól használható applikáció, ilyen például az egyetemi jegyzetelés. Akinek nem makulátlan a kézírása, és a gyors tempójú előadások, gyakorlatok során sietősen kell papírra vetnie az elhangzottakat, az gyakran szembesülhet vele, hogy a következő héten már nem tudja elolvasni az előző heti jegyzetét. Ha pedig valaki tömegközlekedésen szeretné az időt hasznosan eltölteni, vagy utazás közben tanulni, akkor mindig mindenhova vinnie kell magával a füzeteket, illetve könyveket. Ezen a kényelmetlen helyzeten szerettem volna egy kicsit segíteni az alkalmazással. 

A célom az volt, hogy egy hétköznapokban kényelmesen használható programot hozzak létre, mely az optikai karakterfelismerés - optical character recognition, röviden OCR - segítségével a lefotózott dokumentumokat digitalizálja. A projekt megvalósítása során létrehoztam egy Android kliensalkalmazást, mely a Google Cloud Vision API használatával digitális szöveggé alakítja a fényképen megjelenő nyomtatott/írott szöveget, és azt egy tetszőleges struktúrában Firebase segítségével eltárolja és megjeleníti. 

\vfill
\selectenglish


%----------------------------------------------------------------------------
% Abstract in English
%----------------------------------------------------------------------------
\chapter*{Abstract}\addcontentsline{toc}{chapter}{Abstract}

Due to the rapid technical advancement of the past couple of years, smartphones have conquered the world, and as of today a significant portion of our society owns at least one smart device. As a consequence we do almost everything on our phones: keeping in touch, official matters or even studying. In order to make these easier native apps are constantantly being released that simplify the tasks at hand. 

However, there are a lot of use cases where there are no such applications yet, and taking notes at the university is one of them. The person whose handwriting is not perfectly clean and has to quickly jot the material down during lectures and practices often has to face the fact that they cannot read their own handwriting from the week before. And if someone wants to use their time wisely on public transportation, or study a little while travelling, then they always have to take their notebooks and books with them everywhere. These are the main, uncomfortable scenarios that I wanted to help with this application.

My goal was to produce a program, that's comfortable to use in everyday life, which digitizes pictures of documents with the help of optical character recognition, or OCR for short. During the realization I created an Android client application that uses Google Cloud Vision API to produce digital text from a printed/handwritten text in a picture, and stores it in a user-defined structure with the help of Firebase.


\vfill
\selectthesislanguage

\newcounter{romanPage}
\setcounter{romanPage}{\value{page}}
\stepcounter{romanPage}