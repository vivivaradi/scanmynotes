%-----------------------
\chapter{Alkalmazás}
%-----------------------
%TODO 

Egy alkalmazás megvalósítása mindig komplex folyamat, mely sok kisebb részfolyamatot foglal magába. Az alábbiakban a fejlesztés részleteiről fogok írni, kezdve az alkalmazással szemben támasztott követelményekkel, majd képernyőképek segítségével bemutatom a működését. Ezután a megvalósítás részleteibe fogok belemenni, és végül a tesztelésről is ejtek pár szót.

\section{Követelmények}

A fejlesztés során alapvető célom volt egy, a modern paradigmáknak és a felhasználók elvárásainak megfelelő applikáció megalkotása. Törekedtem az objektumorientált szemlélet alkalmazására, a felelősségek szétválasztására és a maximális felhasználói élmény nyújtására. Fontos volt, hogy az alkalmazás folyamatai ne térjenek el nagymértékben attól, mint amit a felhasználók más applikációkban megszokhattak, és minden művelet meg legyen valósítva az adatokon, amire szükségük lehet. 

Ezek alapján az alábbi követelmények fogalmazódtak meg az alkalmazással szemben:
\begin{itemize}
	\item A felhasználónak legyen lehetősége fiókot létrehozni, e-mail cím és jelszó megadásával bejelentkezni, illetve fiókjából ki is lépni.
	\item Az adatok legyenek perzisztensen tárolva, az alkalmazás bezárásával, abból való kilépéssel vagy egy esetleges készülékcsere esetén sem veszhetnek el.
	\item Az adatok csak bejelentkezés után váljanak láthatóvá.
	\item A felhasználó képes legyen kategóriákat létrehozni a jegyzetek számára, ezeket akár tetszőleges mélységben más kategóriákba ágyazni.
	\item Legyen lehetőség jegyzetek létrehozására dokumentumok lefényképezése által
\end{itemize}

\section{Működés}

\section{Megvalósítás}

\section{Tesztelés}