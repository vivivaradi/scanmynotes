\chapter{Követelmények}

A fejlesztés során alapvető célom volt egy, a modern paradigmáknak és a felhasználók elvárásainak megfelelő applikáció megalkotása. Törekedtem az objektumorientált szemlélet alkalmazására, a felelősségek szétválasztására és a maximális felhasználói élmény nyújtására. Fontos volt, hogy az alkalmazás folyamatai ne térjenek el nagymértékben attól, mint amit a felhasználók más applikációkban megszokhattak, és minden művelet meg legyen valósítva az adatokon, amire szükségük lehet.

Ezek alapján az alábbi követelmények fogalmazódtak meg az alkalmazással szemben:
\begin{enumerate}
	\item A felhasználónak legyen lehetősége fiókot létrehozni, e-mail cím és jelszó megadásával bejelentkezni, illetve fiókjából ki is lépni.
	\item Az adatok legyenek perzisztensen tárolva, az alkalmazás bezárásával, abból való kilépéssel vagy egy esetleges készülékcsere esetén se vesszenek el.
	\item Az adatok csak bejelentkezés után váljanak láthatóvá, minden felhasználó csak a saját adataihoz férhessen hozzá.
	\item A felhasználó képes legyen kategóriákat létrehozni a jegyzetek számára, ezeket akár tetszőleges mélységben további kategóriákba ágyazni.
	\item Legyen lehetőség jegyzetek létrehozására dokumentumok lefényképezése által. A szöveg digitalizálása után a jegyzet legyen címmel ellátható, kategóriába sorolható és a tartalma szerkeszthető.
	\item Inkonzisztens adatok létrehozására ne adjon lehetőséget, a felhasználói input mindig legyen validálva.
	\item A jegyzeteken és kategóriákon lehessen minden fő műveletet - létrehozás, olvasás, módosítás, törlés - elvégezni, ezek során a felhasználói felület és az adatbázis maradjanak konzisztensek egymással.
	\item Módosítás során lehessen a jegyzetet újabb fényképek készítésével kiegészíteni, az ily módon digitalizált szöveg kerüljön hozzáfűzésre az eredeti tartalomhoz.
	\item A UI esztétikai és felhasználói élmény szempontjából legyen megfelelő, a hosszú ideig tartó folyamatokat jelezze a felhasználónak töltőképernyő segítségével.
	\item A jegyzetek listája lehessen rendezhető és kereshető.
\end{enumerate}