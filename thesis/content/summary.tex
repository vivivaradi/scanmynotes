%--------------------
\chapter{Összefoglalás}
%--------------------

%TODO

\section{Tapasztalatok}

\section{Értékelés}

Az alkalmazás követelményei résznél 10 pontba szedtem az applikációval szemben támasztott elvárásokat, így most arra szeretnék kitérni pontonként, hogy ezeket mennyire sikerült megvalósítani.

\begin{enumerate}
	\item A felhasználói fiók létrehozására és bejelentkezésre létrejött egy képernyő, ahol egy e-mail címet és egy jelszót kell megadni. Magát az autentikációt a Firebase Authentication végzi. Belépés után a NoteList képernyő jobb felső sarkában található Logout gomb nyújt lehetőséget a kijelentkezésre.
	\item Az adatok tárolását a Firebase Firestore végzi, és mivel ez egy felhőalapú szolgáltatás, így a felhasználó tetszőleges készüléken bejelentkezve eléri a saját adatait. 
	\item A bejelentkezési oldalról addig nem lehet tovább navigálni, amíg nem lett sikeres az autentikáció. A saját adatok elérése pedig az adattárolás struktúrájából következik, ugyanis az adatbázisban minden felhasználó adata a saját azonosítójával elnevezett dokumentumban található, futás során pedig az auth modul csak a jelenleg bejelentkezett felhasználó id-ját ismeri, másét nem. 
	\item 
	\item
	\item
	\item
	\item
	\item
	\item
\end{enumerate}

Mindezt figyelembe véve én elégedett vagyok az elkészült alkalmazással, mert bár még elég alap funkcionalitással rendelkezik, de úgy gondolom, hogy a lehető legjobban figyeltem arra, hogy könnyen bővíthetővé tegyem. Van is nem kevés ötletem a potenciális fejlesztésekre, melyből néhányat a következő szekcióban ismertetek. 

\section{Továbbfejlesztési lehetőségek}

Természetesen az alkalmazás abszolút nem tökéletes, van még rengeteg opció a fejlődésre. Ezért úgy gondoltam, hogy lezárásképp megemlítek pár dolgot, amin lehetne javítani mind kódminőség, mind felhasználás szempontjából.