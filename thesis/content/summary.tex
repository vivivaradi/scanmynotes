%--------------------
\chapter{Összefoglalás}
%--------------------

%TODO intro

\section{Tapasztalatok}

\section{Értékelés}

Az 5. fejezetben az alkalmazás követelményei résznél 10 pontba szedtem az applikációval szemben támasztott elvárásokat, így most arra szeretnék kitérni pontonként, hogy ezeket mennyire sikerült megvalósítani.

\begin{enumerate}
	\item A felhasználói fiók létrehozására és a bejelentkezésre létrejött egy képernyő, ahol egy e-mail címet és egy jelszót kell megadni. Magát az autentikációt a Firebase Authentication végzi. Belépés után a NoteList képernyő jobb felső sarkában található Logout gomb nyújt lehetőséget a kijelentkezésre.
	\item Az adatok tárolásáért a Firebase Firestore a felelős, és mivel ez egy felhőalapú szolgáltatás, így az bárhonnan, bármikor elérhető bejelentkezést követően. Nem kell tehát a felhasználónak azzal foglalkoznia, hogy biztonsági másolatot készítsen a jegyzeteiről arra az esetre, ha elromlana/eltűnne a telefonja, csupán annyit kell tennie, hogy egy másik készülékre letölti az applikációt és bejelentkezik.
	\item A bejelentkezési oldalról addig nem lehet tovább navigálni, amíg nem lett sikeres az autentikáció. A saját adatok elérése pedig az adattárolás struktúrájából következik, ugyanis az adatbázisban minden felhasználó adata a saját azonosítójával elnevezett dokumentumban található, futás során pedig az auth modul csak a jelenleg bejelentkezett felhasználó id-ját ismeri, és ezt használja fel a megfelelő dokumentum lekéréséhez. 
	\item A kategóriákkal szemben támasztott követelményeket sikerült maradéktalanul teljesíteni, van egy új kategória nézet, ahol nevet és szülőt lehet adni a kategóriának. A Category modell pedig biztosítja a tetszőleges mélységű egymásba ágyazást, ugyanis a \emph{listItems} adattagja további kategóriákat is tartalmazhat, melyek szintén tartalmazhatnak kategóriákat és így tovább. 
	\item Egy fénykép készítése után - amennyiben volt rajta felismerhető szöveg - annak tartalmából jegyzet készíthető, a megjelenő létrehozás képernyőn címet adhatunk és kategóriát választhatunk neki, és a detektált szöveg is szerkeszthető mentés előtt. 
	\item Az alkalmazás mindenhol végez validációt, ahol felhasználói szöveges inputot vár: a bejelentkezési adatokon e-mail cím formátumot és minimum karakterszámot ellenőriz, az objektumok létrehozásakor és szerkesztésekor pedig a cím, illetve tartalom ürességét figyeli. 
	\item 
	\item
	\item
	\item
\end{enumerate}

Mindezt figyelembe véve én elégedett vagyok az elkészült alkalmazással, mert bár még elég alap funkcionalitással rendelkezik, de úgy gondolom, hogy a lehető legjobban figyeltem arra, hogy könnyen bővíthetővé tegyem. Van is nem kevés ötletem a potenciális fejlesztésekre, melyből néhányat a következő szekcióban ismertetek. 

\section{Továbbfejlesztési lehetőségek}

Egy alkalmazás fejlesztése soha nem ér véget abból a szempontból, hogy mindig lehet szebbé és jobbá alakítani. A félév végére egy egész komplex projekt alakult ki, de lezárásképp még megemlítek pár dolgot, amiben tovább lehetne fejleszteni.